\documentclass{article}
\usepackage{polski}
\usepackage[utf8]{inputenc}
\usepackage{graphicx}
\usepackage[a4paper, total={7in, 10in}]{geometry}
\usepackage{listings}
\usepackage{amsmath}
\usepackage[section]{placeins}
\graphicspath{ {./images/} }

\title{Zastosowanie Qlearningu w zachowaniu boidów}
\author{Jakub Łęcki, Marek Hering, Maciej Jabłoński}
\date{08.06.2020}
\begin{document}
\maketitle
\begin{abstract}
    Tematem niniejszej pracy jest problem nauczenia boidów (w naszym przypadku są to ryby), aby poprzez prawidłowe poruszanie się maksymalizowały swój czas życia. W tym celu użyliśmy koncepcji qlearningu oraz algorytmu stada.
\end{abstract}

\section{Boidy}
\subsection{Pochodzenie}
Termin \textbf{boid} został stworzony przez Craiga Reynoldsa w 1987 roku jako określenie stworzenia wykazującego cechy stadne. Słowo boid wzięło się z uproszeczenia terminu 'boid-like' jako odniesienie do ptaków formujących się w gromady.
\subsection{Zasady zachowania}
Okazuje się, że w świecie rzeczywistym wiele gatunków zwierząt łączących się w grupy wykazuje podobne własności. Patrząc na stada ptaków, ławice ryb, roje pszczół lub stada owiec można zauważyć, że każda z jednostek stosuje się do 3 podstawowych zasad:
\begin{enumerate}
    \item Rozdzielność - osobnik nie lubi przebywać w tłoku, dlatego zachowuje dystans do swoich sąsiadów
    \item Spójność - osobnik nie lubi przebywać w samotności, więc kieruje się ku najbliższym współstadnikom 
    \item Wyrównanie - osobnik porusza się w kierunku zbliżonym do kierunku otaczających członków stada
\end{enumerate}

Łącząc te 3 proste zasady, boidy tworzą złożone i bardzo zorganizowane skupiska, które obserwujemy jako np. ławice ryb, które pozostają w płynnym, nieustannym ruchu.
\section{Qlearning}
Jest to jedna z technik szerokiej dziedziny uczenia maszynowego znanej jako "Uczenie ze wzmocnieniem" (ang. Reinforcement Learning). Jej dwoma głównymi elementami są tablica stanów-akcji zwykle nazywana jako \textbf{qtable} oraz środowisko, po którym porusza się agent.
\subsection{Qtable}
Tablica ma wymiary \(n x m\), gdzie:
\begin{itemize}
    \item m - liczba możliwych stanów
\end{itemize}


\end{document}